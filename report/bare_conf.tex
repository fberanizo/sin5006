
\documentclass[conference]{IEEEtran}

\usepackage[utf8]{inputenc}
\usepackage[T1]{fontenc}
\usepackage{lmodern} % load a font with all the characters
\usepackage{xcolor}
\usepackage{multicol}
\usepackage{graphicx}
\usepackage{subfig}
\usepackage{afterpage}
\graphicspath{ {images/} }
\newcommand\myworries[1]{\textcolor{red}{#1}}

% correct bad hyphenation here
\hyphenation{op-tical net-works semi-conduc-tor}


\begin{document}

\title{Trabalho de Algorítmos Genéticos\\ Parte 1}


\author{\IEEEauthorblockN{Fábio Beranizo Fontes Lopes}
\IEEEauthorblockA{Escola de Artes, Ciências e Humanidades (EACH)\\
Universidade de São Paulo (USP)\\
Email: f.lopes@usp.br}
}

% make the title area
\maketitle

% As a general rule, do not put math, special symbols or citations
% in the abstract
\begin{abstract}
Este relatório apresenta a discussão da parte 2  do trabalho de algorítmos
genéticos (GA) para a disciplina de Inteligência Computacional. Foi avaliada a
utilização de GA para otimização de um conjunto funções. O foco se manteve em
analisar o comportamento do algorítmo mediante a utilização de
diferentes técnicas, operadores e parametrizações.\\
O trabalho foi desenvolvido utilizando a linguagem de programação Python,
especificamente sua distribuição Anaconda.\\
\end{abstract}

% no keywords


\section{Operadores Clássicos}

% Aplicando operadores clássicos: esses são os operadores da parte I do trabalho. Esses operadores vão gerar muitas soluções infactíveis. Vocês devem analisar essa infactibilidade, mostrando como ela ocorre e por que. Discutam suas observações.
% b) aplicando operadores de correção: esses operadores são novos operadores, que vocês devem criar (há pelo menos um na bibliografia disponibilizada), para corrigir a infactibilidade criada na letra 

% TODO explicar operadores da primeira parte (crossover de um ponto, uniforme)

\subsection{Geração de Instâncias}
Para a geração de uma população inicial válida, foi tomada a seguinte abordagem:
\begin{enumerate}
\item Criar uma ordenação aleatória dos clientes
\item Iniciando do primeiro cliente ordenado, alocar demanda de cada cliente a um veículo
\item Se uma demanda exceder a capacidade restante do veículo, alocar outro veículo
\end{enumerate}

O modelo de genótipo escolhido foi uma lista que guarda a sequencia de clientes servidas por um veículo, e separada por um caracter ``X'' que representa o início do caminho de um outro veículo.

\subsection{Problemas com operadores}
Durante o desenvolvimento da solução, foi detectado que a aplicação dos operadores de mutação e de crossover poderia criar soluções infactíveis para o problema. Para um cromossomo cujos genes representam um caminho, a aplicação de crossover leva fatalmente à geração de caminhos incompletos.\\

Para contornar este problema uma primeira abordagem tomada foi repetir a realização do crossover por quantas vezes quanto necessário até que seja gerado um crossover válido. No entanto após a realização de testes esta possibilidade foi descartada, uma vez que demorava muito até que surgisse uma solução válida. Além disso, muito recurso computacional era disperdiçado, fazendo com que essa solução fosse inviável. Assim, foi necessário buscar alternativas na literatura para resolver o problema.

A primeira solução adotada foi não realizar crossover.

% For peer review papers, you can put extra information on the cover
% page as needed:
% \ifCLASSOPTIONpeerreview
% \begin{center} \bfseries EDICS Category: 3-BBND \end{center}
% \fi
%
% For peerreview papers, this IEEEtran command inserts a page break and
% creates the second title. It will be ignored for other modes.
\IEEEpeerreviewmaketitle



\section{Introduction}
% no \IEEEPARstart
This demo file is intended to serve as a ``starter file''
for IEEE conference papers produced under \LaTeX\ using
IEEEtran.cls version 1.8b and later.
% You must have at least 2 lines in the paragraph with the drop letter
% (should never be an issue)
I wish you the best of success.

\hfill mds
 
\hfill August 26, 2015

\subsection{Subsection Heading Here}
Subsection text here.


\subsubsection{Subsubsection Heading Here}
Subsubsection text here.

% An example of a floating figure using the graphicx package.
% Note that \label must occur AFTER (or within) \caption.
% For figures, \caption should occur after the \includegraphics.
% Note that IEEEtran v1.7 and later has special internal code that
% is designed to preserve the operation of \label within \caption
% even when the captionsoff option is in effect. However, because
% of issues like this, it may be the safest practice to put all your
% \label just after \caption rather than within \caption{}.
%
% Reminder: the "draftcls" or "draftclsnofoot", not "draft", class
% option should be used if it is desired that the figures are to be
% displayed while in draft mode.
%
%\begin{figure}[!t]
%\centering
%\includegraphics[width=2.5in]{myfigure}
% where an .eps filename suffix will be assumed under latex, 
% and a .pdf suffix will be assumed for pdflatex; or what has been declared
% via \DeclareGraphicsExtensions.
%\caption{Simulation results for the network.}
%\label{fig_sim}
%\end{figure}

% Note that the IEEE typically puts floats only at the top, even when this
% results in a large percentage of a column being occupied by floats.


% An example of a double column floating figure using two subfigures.
% (The subfig.sty package must be loaded for this to work.)
% The subfigure \label commands are set within each subfloat command,
% and the \label for the overall figure must come after \caption.
% \hfil is used as a separator to get equal spacing.
% Watch out that the combined width of all the subfigures on a 
% line do not exceed the text width or a line break will occur.
%
%\begin{figure*}[!t]
%\centering
%\subfloat[Case I]{\includegraphics[width=2.5in]{box}%
%\label{fig_first_case}}
%\hfil
%\subfloat[Case II]{\includegraphics[width=2.5in]{box}%
%\label{fig_second_case}}
%\caption{Simulation results for the network.}
%\label{fig_sim}
%\end{figure*}
%
% Note that often IEEE papers with subfigures do not employ subfigure
% captions (using the optional argument to \subfloat[]), but instead will
% reference/describe all of them (a), (b), etc., within the main caption.
% Be aware that for subfig.sty to generate the (a), (b), etc., subfigure
% labels, the optional argument to \subfloat must be present. If a
% subcaption is not desired, just leave its contents blank,
% e.g., \subfloat[].


% An example of a floating table. Note that, for IEEE style tables, the
% \caption command should come BEFORE the table and, given that table
% captions serve much like titles, are usually capitalized except for words
% such as a, an, and, as, at, but, by, for, in, nor, of, on, or, the, to
% and up, which are usually not capitalized unless they are the first or
% last word of the caption. Table text will default to \footnotesize as
% the IEEE normally uses this smaller font for tables.
% The \label must come after \caption as always.
%
%\begin{table}[!t]
%% increase table row spacing, adjust to taste
%\renewcommand{\arraystretch}{1.3}
% if using array.sty, it might be a good idea to tweak the value of
% \extrarowheight as needed to properly center the text within the cells
%\caption{An Example of a Table}
%\label{table_example}
%\centering
%% Some packages, such as MDW tools, offer better commands for making tables
%% than the plain LaTeX2e tabular which is used here.
%\begin{tabular}{|c||c|}
%\hline
%One & Two\\
%\hline
%Three & Four\\
%\hline
%\end{tabular}
%\end{table}


% Note that the IEEE does not put floats in the very first column
% - or typically anywhere on the first page for that matter. Also,
% in-text middle ("here") positioning is typically not used, but it
% is allowed and encouraged for Computer Society conferences (but
% not Computer Society journals). Most IEEE journals/conferences use
% top floats exclusively. 
% Note that, LaTeX2e, unlike IEEE journals/conferences, places
% footnotes above bottom floats. This can be corrected via the
% \fnbelowfloat command of the stfloats package.




\section{Conclusion}
The conclusion goes here.




% conference papers do not normally have an appendix


% use section* for acknowledgment
\section*{Acknowledgment}


The authors would like to thank...





% trigger a \newpage just before the given reference
% number - used to balance the columns on the last page
% adjust value as needed - may need to be readjusted if
% the document is modified later
%\IEEEtriggeratref{8}
% The "triggered" command can be changed if desired:
%\IEEEtriggercmd{\enlargethispage{-5in}}

% references section

% can use a bibliography generated by BibTeX as a .bbl file
% BibTeX documentation can be easily obtained at:
% http://mirror.ctan.org/biblio/bibtex/contrib/doc/
% The IEEEtran BibTeX style support page is at:
% http://www.michaelshell.org/tex/ieeetran/bibtex/
%\bibliographystyle{IEEEtran}
% argument is your BibTeX string definitions and bibliography database(s)
%\bibliography{IEEEabrv,../bib/paper}
%
% <OR> manually copy in the resultant .bbl file
% set second argument of \begin to the number of references
% (used to reserve space for the reference number labels box)
\begin{thebibliography}{1}

\bibitem{IEEEhowto:kopka}
H.~Kopka and P.~W. Daly, \emph{A Guide to \LaTeX}, 3rd~ed.\hskip 1em plus
  0.5em minus 0.4em\relax Harlow, England: Addison-Wesley, 1999.

\end{thebibliography}




% that's all folks
\end{document}


